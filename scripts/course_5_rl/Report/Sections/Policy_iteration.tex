The policy improvement theorem states that a new policy which is greedy with respect to the value function of an existing policy has equal or better value than the existing policy in every state. We start off by proving that a policy $\sigma$ which is greedy with respect to the value of a single state under $\pi$ (and is equal to $\pi$ in every other state) has value in every state which is greater than or equal to the value of the same state under $\pi$, and then apply similar reasoning to prove the full policy improvement theorem. To begin with, say we are using a (possibly stochastic) policy $\pi$, and consider a policy $\sigma$, which is identical to $\pi$ in all states except $\bar{s}$, in which $\sigma$ chooses action $\bar{a}$ deterministically:
\begin{equation*}
    \sigma(a\vert s) = \begin{cases}
        \pi(a\vert s) & s \ne \bar{s} \\
        1 & s=\bar{s}, a=\bar{a} \\
        0 & s=\bar{s}, a\ne\bar{a}
    \end{cases}
\end{equation*}
Is the value of choosing action $\bar{a}$ in state $\bar{s}$ higher under the new policy? We can express the difference in the value of this state-action pair under the two policies in terms of the change in value of every state:
\begin{equation*}
    Q^\sigma(\bar{s}, \bar{a}) - Q^\pi(\bar{s}, \bar{a}) = \gamma \sum_{s'\in\mathcal{S}}{\Bigl[ p(s'\vert \bar{s}, \bar{a}) \Bigl( V^\sigma(s') - V^\pi(s') \Bigr) \Bigr]}
\end{equation*}
The change in value of every state besides $\bar{s}$ can be expressed recursively:
\begin{align*}
    (\forall s \ne \bar{s}) \quad V^\sigma(s) - V^\pi(s) &= \gamma \sum_{a\in\mathcal{A}}{\left[ \pi(a\vert s) \sum_{s'\in\mathcal{S}}{\Bigl[ p(s'\vert s, a) \Bigl( V^\sigma(s') - V^\pi(s') \Bigr) \Bigr]} \right]} \\
    &= \gamma\sum_{s'\in\mathcal{S}}{\left[ \underbrace{\sum_{a\in\mathcal{A}}{\Bigl[ \sigma(a\vert s) p(s'\vert s, a) \Bigr]}}_{P^\sigma_{s, s'}} \Bigl( V^\sigma(s') - V^\pi(s') \Bigr) \right]}
\end{align*}
The change in the value of state $\bar{s}$ can be expressed as follows:
\begin{align*}
    V^\sigma(\bar{s}) &= \sum_{a\in\mathcal{A}}{\Bigl[ \sigma(a\vert \bar{s}) Q^\sigma(\bar{s}, a) \Bigr]} \\
    &= Q^\sigma(\bar{s}, \bar{a}) \\
    \Rightarrow \quad V^\sigma(\bar{s}) - V^\pi(\bar{s}) &= Q^\sigma(\bar{s}, \bar{a}) - V^\pi(\bar{s}) \\
    &= \Bigl(Q^\sigma(\bar{s}, \bar{a}) - Q^\pi(\bar{s}, \bar{a})\Bigr) + \Bigl(Q^\pi(\bar{s}, \bar{a}) - V^\pi(\bar{s})\Bigr) \\
    &= \left( \gamma \sum_{s'\in\mathcal{S}}{\left[ p(s'\vert \bar{s}, \bar{a}) \Bigl( V^\sigma(s') - V^\pi(s') \Bigr) \right]} \right) + \Bigl(Q^\pi(\bar{s}, \bar{a}) - V^\pi(\bar{s})\Bigr) \\
    &= \gamma\sum_{s'\in\mathcal{S}}{\left[ \underbrace{\sum_{a\in\mathcal{A}}{\Bigl[ \sigma(a\vert \bar{s}) p(s'\vert \bar{s}, a) \Bigr]}}_{P^\sigma_{\bar{s}, s'}} \Bigl( V^\sigma(s') - V^\pi(s') \Bigr) \right]} + \Bigl(Q^\pi(\bar{s}, \bar{a}) - V^\pi(\bar{s})\Bigr) \\
\end{align*}
These expressions can now be assembled in matrix form using the basis vector $e_{\bar{s}}$ for state $\bar{s}$:
\begin{align*}
    v^\sigma - v^\pi &= \gamma P^\sigma (v^\sigma - v^\pi) + \Bigl(Q^\pi(\bar{s}, \bar{a}) - V^\pi(\bar{s})\Bigr)e_{\bar{s}} \\
    \text{where} \quad (e_{\bar{s}})_i &= \begin{cases}
        1 & i = \bar{s} \\
        0 & i \ne \bar{s}
    \end{cases} \\
    \Rightarrow \quad v^\sigma - v^\pi &= \gamma P^\sigma \left( \gamma P^\sigma (v^\sigma - v^\pi) + \Bigl(Q^\pi(\bar{s}, \bar{a}) - V^\pi(\bar{s})\Bigr)e_{\bar{s}} \right) + \Bigl(Q^\pi(\bar{s}, \bar{a}) - V^\pi(\bar{s})\Bigr)e_{\bar{s}} \\
    &= \Bigl(Q^\pi(\bar{s}, \bar{a}) - V^\pi(\bar{s})\Bigr) \Bigl( I + \gamma P^\sigma \Bigr) e_{\bar{s}} + \gamma^2 (P^\sigma)^2 (v^\sigma - v^\pi) \\
    &\vdots \\
    &= \Bigl(Q^\pi(\bar{s}, \bar{a}) - V^\pi(\bar{s})\Bigr) \sum_{n=0}^{N-1} \Bigl[ \gamma^n (P^\sigma)^n \Bigr] e_{\bar{s}} + \gamma^N (P^\sigma)^N (v^\sigma - v^\pi) \\
    &= \Bigl(Q^\pi(\bar{s}, \bar{a}) - V^\pi(\bar{s})\Bigr) \sum_{n=0}^{\infty} \Bigl[ \gamma^n (P^\sigma)^n \Bigr] e_{\bar{s}} \\
    (\forall s,s'\in\mathcal{S})(\forall n\in\mathbb{N}) \quad (\gamma^n (P^\sigma)^n)_{s,s'} &\ge 0 \\
    \Rightarrow \quad \Bigl(Q^\pi(\bar{s}, \bar{a}) > V^\pi(\bar{s})\Bigr) &\Rightarrow \Bigl( (\forall s\in\mathcal{S}) \quad V^\sigma(s) \ge V^\pi(s) \Bigr)
\end{align*}
This completes the proof that greedification with respect to a single state $\bar{s}$ leads to policy improvement. We now turn to the full policy improvement theorem, which states that greedification with respect to all states leads to policy improvement. Consider the same original policy $\pi$, and a new policy $\sigma$ which in every state $s$ chooses action $a$ which is greedy with respect to $Q^\pi(s, a)$ (ties in the $\operatorname{argmax}$ operator are broken arbitrarily):
\begin{equation*}
    (\forall s\in\mathcal{S})(\forall a\in\mathcal{A})\quad\sigma(a\vert s) = \begin{cases}
        1 & a = \underset{a'}{\operatorname{argmax}}\Bigl[ Q^\pi(s, a') \Bigr] \\
        0 & \text{otherwise}
    \end{cases}
\end{equation*}
Denote the action chosen by $\sigma$ in state $s$ by $\bar{a}_s$:
\begin{equation*}
    (\forall s\in\mathcal{S})\quad \bar{a}_s = \underset{a'}{\operatorname{argmax}}\Bigl[ Q^\pi(s, a') \Bigr]
\end{equation*}
The difference in the state-action value function of choosing action $\bar{a}_s$ in state $s$ under policies $\sigma$ and $\pi$ can be expressed in terms of the differences in state value functions of subsequent states:
\begin{equation*}
    Q^\sigma(s, \bar{a}_s) - Q^\pi(s, \bar{a}_s) = \gamma\sum_{s'\in\mathcal{S}}{\left[ p(s'\vert s, \bar{a}_s) \Bigl( V^\sigma(s') - V^\pi(s')\Bigr) \right]}
\end{equation*}
Note that the state-action value function $Q^\sigma(s, \bar{a}_s)$ of $\sigma$ choosing action $\bar{a}_s$ in state $s$ is equal to the state value function $V^\sigma(s)$ of $s$ under $\sigma$ (because $\sigma$ chooses $\bar{a}_s$ deterministically):
\begin{align*}
    V^\sigma(s) &= \sum_{a\in\mathcal{A}}{\Bigl[ \sigma(a\vert s) Q^\sigma(s, a) \Bigr]} \\
    &= Q^\sigma(s, \bar{a}_s)
\end{align*}
Also note that the state-action value function $Q^\pi(s, \bar{a}_s)$ of $\pi$ choosing action $\bar{a}_s$ in state $s$ is related to the state value function $V^\pi(s)$ of $s$ under $\pi$ by the advantage function $A^\pi(s, \bar{a}_s)$, which in this case is necessarily non-negative:
\begin{align*}
    (\forall s\in\mathcal{S})\quad V^\pi(s) &= \sum_{a\in\mathcal{A}}{\Bigl[ \pi(a\vert s) Q^\pi(s, a) \Bigr]} \\
    &\le \sum_{a\in\mathcal{A}}{\left[ \pi(a\vert s) \underset{a'}{\operatorname{max}}\Bigl[ Q^\pi(s, a') \Bigr] \right]} \\
    &= \underset{a'}{\operatorname{max}}\Bigl[ Q^\pi(s, a') \Bigr] \\
    &= Q^\pi(s, \bar{a}_s) \\
    \Rightarrow\quad V^\pi(s) &\le Q^\pi(s, \bar{a}_s) \\
    A^\pi(s, \bar{a}_s) &= Q^\pi(s, \bar{a}_s) - V^\pi(s) \\
    \Rightarrow(\forall s\in\mathcal{S})\quad A^\pi(s, \bar{a}_s) &\ge 0
\end{align*}
The equation for the difference in state-action value functions can now be expressed purely in terms of differences between state value functions and the advantage function, and this equation can further be expressed as a recursive matrix equation:
\begin{align*}
    V^\sigma(s) - \Bigl(V^\pi(s) + A^\pi(s, \bar{a}_s)\Bigr) &= \gamma\sum_{s'\in\mathcal{S}}{\left[ p(s'\vert s, \bar{a}_s) \Bigl( V^\sigma(s') - V^\pi(s')\Bigr) \right]} \\
    \Rightarrow\quad V^\sigma(s) - V^\pi(s) &= A^\pi(s, \bar{a}_s) + \gamma\sum_{s'\in\mathcal{S}}{\left[ p(s'\vert s, \bar{a}_s) \Bigl( V^\sigma(s') - V^\pi(s')\Bigr) \right]} \\
    \Rightarrow\quad v^\sigma - v^\pi &= \beta + \gamma P^\sigma(v^\sigma - v^\pi) \\
    \text{where} \quad & \begin{cases}
        v^\sigma_s &= V^\sigma(s) \\
        v^\pi_s &= V^\pi(s) \\
        \beta_s &= A^\pi(s, \bar{a}_s) \\
        P^\sigma_{s, s'} &= p(s'\vert s, \bar{a}_s)
    \end{cases} \\
    \Rightarrow\quad v^\sigma - v^\pi &= \beta + \gamma P^\sigma\left(\beta + \gamma P^\sigma(v^\sigma - v^\pi)\right) \\
    &= (I + \gamma P^\sigma)\beta + \gamma^2 (P^\sigma)^2(v^\sigma - v^\pi) \\
    &\vdots \\
    &= \sum_{n=0}^{N-1} \Bigl[ \gamma^n (P^\sigma)^n \Bigr] \beta + \gamma^N (P^\sigma)^N (v^\sigma - v^\pi) \\
    &= \sum_{n=0}^{\infty} \Bigl[ \gamma^n (P^\sigma)^n \Bigr] \beta \\
    (\forall s,s'\in\mathcal{S})(\forall n\in\mathbb{N}) \quad (\gamma^n (P^\sigma)^n)_{s,s'} &\ge 0 \\
    (\forall s\in\mathcal{S}) \quad \beta_s &\ge 0 \\
    \Rightarrow (\forall s\in\mathcal{S}) \quad \left(\sum_{n=0}^{\infty} \Bigl[ \gamma^n (P^\sigma)^n \Bigr] \beta\right)_s &\ge 0 \\
    \Rightarrow (\forall s\in\mathcal{S}) \quad V^\sigma(s) &\ge V^\pi(s)
\end{align*}
This concludes the proof of the policy improvement theorem.

Value iteration is a dynamic programming approach which attempts to find the optimal state value function using the following iterative update:
\begin{align*}
    V_{k+1}(s) &= \max_a{\left[ r(s, a) + \gamma\sum_{s'\in\mathcal{S}}{\Bigl[ p(s'\vert s, a) V_k(s') \Bigr]} \right]}
\end{align*}
It should be possible to prove that value iteration converges to the optimal state value function using a combination of the arguments used to prove the convergence of policy iteration and the policy improvement theorem, with actions being selected so as to maximise the current \emph{estimate} of the state-action value functions, which includes some unknown error $\varepsilon_k$. A further challenge in proving that value iteration converges to the optimal state value function is proving that there is a \emph{unique} state value function which is a fixed point of value iteration (which is to say that there is no state value function which is locally optimal across all states but globally sub-optimal).

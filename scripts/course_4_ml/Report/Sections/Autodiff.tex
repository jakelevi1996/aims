Automatic differentiation is essentially equivalent to repeated application of the (multi-dimensional) chain rule to all nodes in a computation graph with respect to their parents, which can be expressed as the following equation (where $P(x)$ is the set of parents of node $x$ in the computation graph):
\begin{equation}
    \frac{\partial f}{\partial x} = \sum_{y\in P(x)}{\left[ \frac{\partial f}{\partial y}\frac{\partial y}{\partial x}\right]}
\end{equation}
For example, the following computation graphs represent calculating gradients of a function of a sum of functions of a variable $x$ (represented by $f(x) = g\Bigl(h(x) + m(x)\Bigr)$) and calculating gradients of a function of a product of functions of a variable $x$ (represented by $f(x) = g\Bigl(h(x) m(x)\Bigr)$) respectively:
\begin{equation*}
    \begin{matrix}
    \begin{matrix}
        \boxed{\begin{matrix}
            y_1=x \\
            \color{red}{\frac{\partial f}{\partial y_1} = \frac{\partial f}{\partial y_2}\frac{\partial y_2}{\partial y_1} + \frac{\partial f}{\partial y_3}\frac{\partial y_3}{\partial y_1}} \\
            \color{red}{= g'(y_4)\Bigl(h'(y_1) + m'(y_1)\Bigr)} \\
            \color{red}{= g'\Bigl(h(x) + m(x)\Bigr)\Bigl(h'(x) + m'(x)\Bigr)}
        \end{matrix}} \\
        \begin{matrix}
            \swarrow & \searrow \\
            \boxed{\begin{matrix}
                y_2=h(y_1) \\
                =h(x) \\
                \color{red}{\frac{\partial f}{\partial y_2} = \frac{\partial f}{\partial y_4}\frac{\partial y_4}{\partial y_2}} \\
                \color{red}{= g'(y_4)}
            \end{matrix}} \qquad & \qquad
            \boxed{\begin{matrix}
                y_3=m(y_1) \\
                =m(x) \\
                \color{red}{\frac{\partial f}{\partial y_3} = \frac{\partial f}{\partial y_4}\frac{\partial y_4}{\partial y_3}} \\
                \color{red}{= g'(y_4)}
            \end{matrix}} \\
            \searrow & \swarrow \\
        \end{matrix} \\
        \boxed{\begin{matrix}
            y_4 = y_2+y_3 \\
            = h(x) + m(x) \\
            \color{red}{\frac{\partial f}{\partial y_4} = \frac{\partial f}{\partial y_5}\frac{\partial y_5}{\partial y_4}} \\
            \color{red}{= g'(y_4)}
        \end{matrix}} \\
        \downarrow \\
        \boxed{\begin{matrix}
            f(x) = y_5 \\
            = g(y_4) \\
            = g\Bigl(h(x) + m(x)\Bigr) \\
            \color{red}{\frac{\partial f}{\partial y_5} = 1}
        \end{matrix}}
    \end{matrix} \qquad & \qquad
    \begin{matrix}
        \boxed{\begin{matrix}
            y_1=x \\
            \color{red}{\frac{\partial f}{\partial y_1} = \frac{\partial f}{\partial y_2}\frac{\partial y_2}{\partial y_1} + \frac{\partial f}{\partial y_3}\frac{\partial y_3}{\partial y_1}} \\
            \color{red}{= g'(y_4)\Bigl(y_3 h'(y_1) + y_2 m'(y_1)\Bigr)} \\
            \color{red}{= g'\Bigl(h(x) m(x)\Bigr)\Bigl(m(x)h'(x) + h(x)m'(x)\Bigr)}
        \end{matrix}} \\
        \begin{matrix}
            \swarrow & \searrow \\
            \boxed{\begin{matrix}
                y_2=h(y_1) \\
                =h(x) \\
                \color{red}{\frac{\partial f}{\partial y_2} = \frac{\partial f}{\partial y_4}\frac{\partial y_4}{\partial y_2}} \\
                \color{red}{= g'(y_4)y_3}
            \end{matrix}} \qquad & \qquad
            \boxed{\begin{matrix}
                y_3=m(y_1) \\
                =m(x) \\
                \color{red}{\frac{\partial f}{\partial y_3} = \frac{\partial f}{\partial y_4}\frac{\partial y_4}{\partial y_3}} \\
                \color{red}{= g'(y_4)y_2}
            \end{matrix}} \\
            \searrow & \swarrow \\
        \end{matrix} \\
        \boxed{\begin{matrix}
            y_4 = y_2 y_3 \\
            = h(x) m(x) \\
            \color{red}{\frac{\partial f}{\partial y_4} = \frac{\partial f}{\partial y_5}\frac{\partial y_5}{\partial y_4}} \\
            \color{red}{= g'(y_4)}
        \end{matrix}} \\
        \downarrow \\
        \boxed{\begin{matrix}
            f(x) = y_5 \\
            = g(y_4) \\
            = g\Bigl(h(x) m(x)\Bigr) \\
            \color{red}{\frac{\partial f}{\partial y_5} = 1}
        \end{matrix}}
    \end{matrix}
    \end{matrix}
\end{equation*}
In practise, this can be applied in three steps. The first step is to set the gradient $\frac{\partial f}{\partial f}$ of the root node $f$ to one and the gradients $\frac{\partial f}{\partial x_i}$ of all other nodes $x_i$ to zero. The second step is to topoligically sort the nodes in the graph (for example using a depth-first search \cite{cormen2001section}) such that any node only appears after all of its parents. The third step is to iterate through every node $y$ in the topologically sorted sequence of nodes, and for every node $y$ iterate through every child node $x$ of $y$ and add $\frac{\partial f}{\partial y}\frac{\partial y}{\partial x}$ to the gradient $\frac{\partial f}{\partial x}$, until the whole graph has been iterated over, at which point the gradient of every node will be correct.

The following computation graphs demonstrate what happens when the function $m(x)$ is replaced by a constant $k$:
\begin{equation*}
    \begin{matrix}
    \begin{matrix}
        \boxed{\begin{matrix}
            y_1=x \\
            \color{red}{\frac{\partial f}{\partial y_1} = \frac{\partial f}{\partial y_2}\frac{\partial y_2}{\partial y_1}} \\
            \color{red}{= g'(y_4)h'(y_1)} \\
            \color{red}{= g'\Bigl(h(x) + k\Bigr)h'(x)}
        \end{matrix}} \\
        \begin{matrix}
            \swarrow \\
            \boxed{\begin{matrix}
                y_2=h(y_1) \\
                =h(x) \\
                \color{red}{\frac{\partial f}{\partial y_2} = \frac{\partial f}{\partial y_4}\frac{\partial y_4}{\partial y_2}} \\
                \color{red}{= g'(y_4)}
            \end{matrix}} \qquad & \qquad
            \boxed{\begin{matrix}
                y_3=k \\
                \color{red}{\frac{\partial f}{\partial y_3} = \frac{\partial f}{\partial y_4}\frac{\partial y_4}{\partial y_3}} \\
                \color{red}{= g'(y_4)} \\
                \color{red}{= g'\Bigl(h(x) + k\Bigr)}
            \end{matrix}} \\
            \searrow & \swarrow \\
        \end{matrix} \\
        \boxed{\begin{matrix}
            y_4 = y_2+y_3 \\
            = h(x) + k \\
            \color{red}{\frac{\partial f}{\partial y_4} = \frac{\partial f}{\partial y_5}\frac{\partial y_5}{\partial y_4}} \\
            \color{red}{= g'(y_4)}
        \end{matrix}} \\
        \downarrow \\
        \boxed{\begin{matrix}
            f(x) = y_5 \\
            = g(y_4) \\
            = g\Bigl(h(x) + k\Bigr) \\
            \color{red}{\frac{\partial f}{\partial y_5} = 1}
        \end{matrix}}
    \end{matrix} \qquad & \qquad
    \begin{matrix}
        \boxed{\begin{matrix}
            y_1=x \\
            \color{red}{\frac{\partial f}{\partial y_1} = \frac{\partial f}{\partial y_2}\frac{\partial y_2}{\partial y_1}} \\
            \color{red}{= g'(y_4)y_3 h'(y_1)} \\
            \color{red}{= g'\Bigl(kh(x)\Bigr)kh'(x)}
        \end{matrix}} \\
        \begin{matrix}
            \swarrow \\
            \boxed{\begin{matrix}
                y_2=h(y_1) \\
                =h(x) \\
                \color{red}{\frac{\partial f}{\partial y_2} = \frac{\partial f}{\partial y_4}\frac{\partial y_4}{\partial y_2}} \\
                \color{red}{= g'(y_4)y_3}
            \end{matrix}} \qquad & \qquad
            \boxed{\begin{matrix}
                y_3=k \\
                \color{red}{\frac{\partial f}{\partial y_3} = \frac{\partial f}{\partial y_4}\frac{\partial y_4}{\partial y_3}} \\
                \color{red}{= g'(y_4)y_2} \\
                \color{red}{= g'\Bigl(k h(x)\Bigr)h(x)}
            \end{matrix}} \\
            \searrow & \swarrow \\
        \end{matrix} \\
        \boxed{\begin{matrix}
            y_4 = y_2 y_3 \\
            = k h(x) \\
            \color{red}{\frac{\partial f}{\partial y_4} = \frac{\partial f}{\partial y_5}\frac{\partial y_5}{\partial y_4}} \\
            \color{red}{= g'(y_4)}
        \end{matrix}} \\
        \downarrow \\
        \boxed{\begin{matrix}
            f(x) = y_5 \\
            = g(y_4) \\
            = g\Bigl(k h(x) \Bigr) \\
            \color{red}{\frac{\partial f}{\partial y_5} = 1}
        \end{matrix}}
    \end{matrix}
    \end{matrix}
\end{equation*}

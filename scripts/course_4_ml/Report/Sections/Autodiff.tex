Automatic differentiation is essentially equivalent to repeated application of the (multi-dimensional) chain rule to all nodes in the graph with respect to their parents, which can be expressed as the following equation (where $P(x)$ is the set of parents of node x in the computational graph):
\begin{equation}
    \frac{\partial f}{\partial x} = \sum_{y\in P(x)}{\left[ \frac{\partial f}{\partial y}\frac{\partial y}{\partial x}\right]}
\end{equation}
For example, the following computational graph represents calculating gradients of a function of a sum of functions of a variable $x$:
\begin{equation*}
    \begin{matrix}
        \boxed{\begin{matrix}
            y_1=x \\
            \color{red}{\frac{\partial f}{\partial y_1} = \frac{\partial f}{\partial y_2}\frac{\partial y_2}{\partial y_1}} \\
            \color{red}{= t'(y_6)s'(y_5)\bigl( q'(y_2) + r'(y_2) \bigr)p'(y_1)}
        \end{matrix}} \\
        \downarrow \\
        \boxed{\begin{matrix}
            y_2=p(y_1) \\
            =p(x) \\
            \color{red}{\frac{\partial f}{\partial y_2} = \frac{\partial f}{\partial y_3}\frac{\partial y_3}{\partial y_2} + \frac{\partial f}{\partial y_4}\frac{\partial y_4}{\partial y_2}} \\
            \color{red}{= t'(y_6)s'(y_5)\bigl( q'(y_2) + r'(y_2) \bigr)}
        \end{matrix}} \\
        \begin{matrix}
            \swarrow & \searrow \\
            \boxed{\begin{matrix}
                y_3=q(y_2) \\
                =q(p(x)) \\
                \color{red}{\frac{\partial f}{\partial y_3} = \frac{\partial f}{\partial y_5}\frac{\partial y_5}{\partial y_3}} \\
                \color{red}{= t'(y_6)s'(y_5)}
            \end{matrix}} \qquad & \qquad
            \boxed{\begin{matrix}
                y_4=r(y_2) \\
                =r(p(x)) \\
                \color{red}{\frac{\partial f}{\partial y_4} = \frac{\partial f}{\partial y_5}\frac{\partial y_5}{\partial y_4}} \\
                \color{red}{= t'(y_6)s'(y_5)}
            \end{matrix}} \\
            \searrow & \swarrow \\
        \end{matrix} \\
        \boxed{\begin{matrix}
            y_5 = y_3+y_4 \\
            = q\bigl(p(x)) + r(p(x)\bigr) \\
            \color{red}{\frac{\partial f}{\partial y_5} = \frac{\partial f}{\partial y_6}\frac{\partial y_6}{\partial y_5}} \\
            \color{red}{= t'(y_6)s'(y_5)}
        \end{matrix}} \\
        \downarrow \\
        \boxed{\begin{matrix}
            y_6 = s(y_5) \\
            = s\biggl(q\bigl(p(x)) + r(p(x)\bigr)\biggr) \\
            \color{red}{\frac{\partial f}{\partial y_6} = \frac{\partial f}{\partial y_7}\frac{\partial y_7}{\partial y_6}} \\
            \color{red}{= t'(y_6)}
        \end{matrix}} \\
        \downarrow \\
        \boxed{\begin{matrix}
            f(x) = y_7 \\
            = t(y_6) \\
            = t\biggl(s\Bigl(q\bigl(p(x)) + r(p(x)\bigr)\Bigr)\biggr) \\
            \color{red}{\frac{\partial f}{\partial y_7} = 1}
        \end{matrix}}
    \end{matrix}
\end{equation*}
In practise, this can be applied by setting the gradient $\frac{\partial f}{\partial f}$ of the root node $f$ to one and the gradients $\frac{\partial f}{\partial x_i}$ of all other nodes $x_i$ to zero, topoligically sorting the nodes in the graph (for example using a depth-first search \cite{cormen2001section}) such that any node only appears after all of its parents, and then iterating through every node $y$ in the topologically sorted sequence of nodes, iterating through every child node $x$ of $y$ and adding $\frac{\partial f}{\partial y}\frac{\partial y}{\partial x}$ to the gradient $\frac{\partial f}{\partial x}$, until the whole graph has been iterated over, at which point the gradient of every node will be correct.

\begin{figure}
    \centering
    \begin{subfigure}{0.45\textwidth}
        \centering
        \includegraphics[width=\textwidth]{MNIST_cross_entropy_loss_over_5_epochs_vs_time,_CPU_vs_GPU,_2_hidden_layers,_400_hidden_units.png}
        \caption{Small model (2 hidden layers, 400 hidden units)}
        \label{fig:mnist small}
    \end{subfigure}
    \begin{subfigure}{0.45\textwidth}
        \centering
        \includegraphics[width=\textwidth]{MNIST_cross_entropy_loss_over_5_epochs_vs_time,_CPU_vs_GPU,_4_hidden_layers,_800_hidden_units.png}
        \caption{Larger model (4 hidden layers, 800 hidden units)}
        \label{fig:mnist larger}
    \end{subfigure}
    \caption{Learning curves for different sized models trained on MNIST over 5 epochs, comparing training times between an Intel(R) Core(TM) i7-1065G7 CPU @ 1.30GHz and an NVIDIA GeForce MX250 GPU. Gaps in the training curves are due to evaluating test set accuracy once per epoch.}
    \label{fig:mnist cpu gpu}
\end{figure}
\begin{figure}
    \centering
    \begin{subfigure}{0.45\textwidth}
        \centering
        \includegraphics[width=\textwidth]{server_MNIST_cross_entropy_loss_over_5_epochs_vs_time,_CPU_vs_GPU,_2_hidden_layers,_400_hidden_units.png}
        \caption{Small model (2 hidden layers, 400 hidden units)}
        \label{fig:mnist small server}
    \end{subfigure}
    \begin{subfigure}{0.45\textwidth}
        \centering
        \includegraphics[width=\textwidth]{server_MNIST_cross_entropy_loss_over_5_epochs_vs_time,_CPU_vs_GPU,_4_hidden_layers,_800_hidden_units.png}
        \caption{Larger model (4 hidden layers, 800 hidden units)}
        \label{fig:mnist larger server}
    \end{subfigure}
    \caption{As for figure \ref{fig:mnist cpu gpu}, performed on a server with Intel(R) Xeon(R) Gold 5120 CPU @ 2.20GHz and NVIDIA TITAN V GPU}
    \label{fig:mnist cpu gpu server}
\end{figure}
\begin{figure}
    \centering
    \begin{subfigure}{0.45\textwidth}
        \centering
        \includegraphics[width=\textwidth]{MNIST_cross_entropy_loss_over_5_epochs_vs_time,_comparing_momentum_parameters.png}
        \caption{Comparing momentum optimisation hyperparameter}
        \label{fig:mnist momentum}
    \end{subfigure}
    \begin{subfigure}{0.45\textwidth}
        \centering
        \includegraphics[width=\textwidth]{MNIST_cross_entropy_loss_over_5_epochs_vs_time,_comparing_batch_size.png}
        \caption{Comparing batch size}
        \label{fig:mnist batch size}
    \end{subfigure}
    \newline
    \begin{subfigure}{0.45\textwidth}
        \centering
        \includegraphics[width=\textwidth]{MNIST_cross_entropy_loss_over_5_epochs_vs_time,_comparing_number_of_hidden_layers.png}
        \caption{Comparing number of hidden layers}
        \label{fig:mnist num hidden layers}
    \end{subfigure}
    \begin{subfigure}{0.45\textwidth}
        \centering
        \includegraphics[width=\textwidth]{MNIST_cross_entropy_loss_over_5_epochs_vs_time,_comparing_dimension_of_hidden_layers.png}
        \caption{Comparing dimension of hidden layers}
        \label{fig:mnist hidden dimension}
    \end{subfigure}
    \newline
    \begin{subfigure}{0.45\textwidth}
        \centering
        \includegraphics[width=\textwidth]{MNIST_cross_entropy_loss_over_5_epochs_vs_time,_comparing_hidden_activation_functions.png}
        \caption{Comparing hidden layer activation functions}
        \label{fig:mnist activation function}
    \end{subfigure}
    \caption{Comparing the effect of different hyperparameters on training a MLP on MNIST. Test set prediction accuracies are included in legends.}
    \label{fig:mnist parameters}
\end{figure}
\begin{figure}
    \centering
    \includegraphics[width=0.6\textwidth]{Test_set_predictions.png}
    \caption{Predictions of a MLP trained on MNIST for 5 epochs on unseen examples of each digit from the test set}
    \label{fig:mnist predictions}
\end{figure}
\begin{figure}
    \centering
    \begin{subfigure}{0.6\textwidth}
        \centering
        \includegraphics[width=\textwidth]{Adversarial_loss_vs_iteration,_5000_iterations,_maximum_pixel_perturbation___0.100.png}
        \caption{Adversarial loss (equation \ref{eq:adversarial problem}) vs iteration}
        \label{fig:adversarial loss}
    \end{subfigure}
    \newline
    \begin{subfigure}{\textwidth}
        \centering
        \includegraphics[width=\textwidth]{Test_set_predictions_with_adversarial_example.png}
        \caption{Trained MLP predictions for adversarial example}
        \label{fig:adversarial prediction}
    \end{subfigure}
    \caption{Training curve and predictions for an adversarial example}
    \label{fig:adversarial example}
\end{figure}
\begin{figure}
    \centering
    \begin{subfigure}{0.45\textwidth}
        \centering
        \includegraphics[width=\textwidth]{overnight_Shakespeare_RNN_training_curve.png}
        \caption{RNN}
        \label{fig:Shakespeare RNN}
    \end{subfigure}
    \begin{subfigure}{0.45\textwidth}
        \centering
        \includegraphics[width=\textwidth]{overnight_Shakespeare_LSTM_training_curve.png}
        \caption{LSTM}
        \label{fig:Shakespeare LSTM}
    \end{subfigure}
    \caption{Loss functions over time while training different sequence models on the complete works of Shakespeare, to predict the next character given a string of training data}
    \label{fig:Shakespeare}
\end{figure}

In this report, Gaussian Processes (GPs) were introduced, and expressions for the GP joint and predictive distributions and the log marginal likelihood (LML) were stated. The Sotonmet dataset of meteorological data was also introduced, in particular the data for tide height as a function of time, shown in figure \ref{fig:data}, which was used to investigate the properties of regression with GPs. Three GPs with squared exponential kernel were considered and evaluated according to different metrics, and the problem of overfitting was discussed, as was the sensitivity of the LML to different hyperparameters. Epistemic and aleatoric uncertainty were discussed, and it was found that a simple GP with optimised hyperparameters performs well under epistemic uncertainty, but not under aleatoric uncertainty, which could be problematic in a safety-critical prediction context. Two GPs with periodic kernels were considered and evaluated, and the tendency of the periodic kernel to assume high certainty about underlying variables and also to assume high observation noise was interpreted in terms of epistemic uncertainty. The problem of optimising the LML with respect to the period of the kernel (to which the LML is very sensitive) was considered, as was the possibility of applying Bayesian optimisation to this problem. Sum and product kernels were discussed, two GPs with each type of kernel were considered and evaluated, and it was found that a GP whose kernel function was the sum of a squared exponential kernel and a periodic kernel achieved the best LML, as well as the qualitatively most plausible looking predictive distribution. Lastly sequential prediction was discussed, in the two cases of prediction with a fixed lookahead period, and prediction using a fixed time-bounded subset of the training data.

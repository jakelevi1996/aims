\subsection{Projection Onto The L1 Ball}
A point $a \in \mathbb{R}^N$ can be projected onto the unit ball in the $\ell_1$ norm by solving the following quadratic program:
\begin{align*}
    \underset{x}{\text{Minimise}} \quad & \frac{1}{2}\Vert x - a \Vert_2^2 \\
    \text{Subject to} \quad & \Vert x \Vert_1 \le c
\end{align*}
This is equivalent to the following quadratic program with slack variables $t$ introduced:
\begin{align*}
    \underset{x,t}{\text{Minimise}} \quad & \frac{1}{2}(x - a)^T(x - a) \\
    \text{Subject to} \quad & x \le t \\
    & x \ge -t \\
    & \mathbf{1}^T t \le c
\end{align*}
The Lagrangian and Lagrangian dual functions can be defined for this problem as follows:
\begin{align*}
    \mathcal{L}(x, t, \lambda) &= \frac{1}{2}(x - a)^T(x - a) + \lambda_1^T(x - t) + \lambda_2^T(-x - t) + \lambda_3(\mathbf{1}^T t - c) \\
    &= \frac{1}{2}x^Tx + (-a + \lambda_1 - \lambda_2)^T x + \frac{1}{2}a^Ta + (-\lambda_1 - \lambda_2 + \lambda_3\mathbf{1})^Tt - \lambda_3c \\
    \frac{\partial\mathcal{L}}{\partial x} &= x + (-a + \lambda_1 - \lambda_2) \\
    \frac{\partial\mathcal{L}}{\partial x} = 0 \quad &\Rightarrow \quad x = a - \lambda_1 + \lambda_2 \\
    &\Rightarrow \quad \mathcal{L}(x, t, \lambda) = -\frac{1}{2}(a - \lambda_1 + \lambda_2)^T(a - \lambda_1 + \lambda_2) + \frac{1}{2}a^Ta + (-\lambda_1 - \lambda_2 + \lambda_3\mathbf{1})^Tt - \lambda_3c \\
    g(\lambda) &= \underset{x, t}{\inf}\left[\mathcal{L}(x, t, \lambda)\right] \\
    &= \begin{cases}
        -\frac{1}{2}(a - \lambda_1 + \lambda_2)^T(a - \lambda_1 + \lambda_2) + \frac{1}{2}a^Ta - \lambda_3c \quad & -\lambda_1 - \lambda_2 + \lambda_3\mathbf{1} = 0 \\
        -\infty & \text{Otherwise}
    \end{cases}
\end{align*}
The dual problem in this case is equivalent to maximising the dual function $g(\lambda)$ with respect to $\lambda\ge0$, which can be expressed as the following quadratic program:
\begin{align*}
    \underset{\lambda_1, \lambda_2, \lambda_3}{\text{Maximise}} \quad & -\frac{1}{2}(a - \lambda_1 + \lambda_2)^T(a - \lambda_1 + \lambda_2) + \frac{1}{2}a^Ta - \lambda_3c \\
    \text{Subject to} \quad & -\lambda_1 - \lambda_2 + \lambda_3\mathbf{1} = 0 \\
    & \lambda_1 \ge 0 \\
    & \lambda_2 \ge 0 \\
    & \lambda_3 \ge 0
\end{align*}
The dual problem could be solved efficiently by using an interior point method, for example. From the solution to the dual problem, the optimal value for $x$ can be found easily as the value that minimises the Lagrangian function:
\begin{equation*}
    x = a - \lambda_1 + \lambda_2
\end{equation*}

\subsection{Svm Duality}
\subsection{Adjustable Optimization}

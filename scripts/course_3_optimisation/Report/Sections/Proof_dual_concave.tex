We start by proving equation \ref{eq:dual concave lambda}, for which it is useful to define the variable $x^*\in\bar{\mathcal{X}}$ (where $\bar{\mathcal{X}}$ denotes the closure of the set $\mathcal{X}$) which satisfies the following equation, given $\alpha$, $\lambda^{(1)}$, $\lambda^{(2)}$, and $\nu$:
\begin{align}
    \underset{x\in\mathcal{X}}{\inf}\left[\mathcal{L}(x, \alpha\lambda^{(1)} + (1 - \alpha)\lambda^{(2)}, \nu)\right] &= \mathcal{L}(x^*, \alpha\lambda^{(1)} + (1 - \alpha)\lambda^{(2)}, \nu) \\
    \Rightarrow & \begin{cases}
        \underset{x\in\mathcal{X}}{\inf}\left[\mathcal{L}(x, \lambda^{(1)}, \nu)\right] \le \mathcal{L}(x^*, \lambda^{(1)}, \nu) \\
        \underset{x\in\mathcal{X}}{\inf}\left[\mathcal{L}(x, \lambda^{(2)}, \nu)\right] \le \mathcal{L}(x^*, \lambda^{(2)}, \nu)
    \end{cases} \\
    \Rightarrow \alpha \underset{x\in\mathcal{X}}{\inf}\left[\mathcal{L}(x, \lambda^{(1)}, \nu)\right] + (1 - \alpha) \underset{x\in\mathcal{X}}{\inf}\left[\mathcal{L}(x, \lambda^{(2)}, \nu)\right] &\le \alpha\mathcal{L}(x^*, \lambda^{(1)}, \nu) + (1 - \alpha)\mathcal{L}(x^*, \lambda^{(2)}, \nu) \\
    &= \mathcal{L}(x^*, \alpha\lambda^{(1)} + (1 - \alpha)\lambda^{(2)}, \nu) \label{eq:follows from Lagrangian} \\
    &= \underset{x\in\mathcal{X}}{\inf}\left[\mathcal{L}(x, \alpha\lambda^{(1)} + (1 - \alpha)\lambda^{(2)}, \nu)\right] \\
    \Rightarrow (\forall\lambda^{(1)},\lambda^{(2)}\in\mathbb{R}^m)(\forall\alpha\in[0, 1]) \quad \alpha g(\lambda^{(1)}, \nu) + (1 - \alpha) g(\lambda^{(2)}, \nu) &\le g(\alpha\lambda^{(1)} + (1 - \alpha)\lambda^{(2)}, \nu) \label{eq:follows from dual}
\end{align}
Where (\ref{eq:follows from Lagrangian}) follows from the definition of the Lagrangian function $\mathcal{L}$ in equation \ref{eq:Lagrangian} and the fact that $\alpha + (1 - \alpha) = 1$, and (\ref{eq:follows from dual}) follows from the definition of the Lagrangian dual function $g$ in equation \ref{eq:dual function}. The proof that the dual Lagrangian function is concave with respect to $\nu$ (equation \ref{eq:dual concave nu}) follows using similar reasoning.

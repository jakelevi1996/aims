\subsection{Standard Form Of Convex Problems}
A convex optimisation problem is one in which the objective function $f_0(x)$ (defined in equation \ref{eq:optimisation problem}) is a convex function, and the feasible set $\mathcal{F}$ (defined in equation \ref{eq:feasible set}) is a convex set. In order for $\mathcal{F}$ to be a convex set, any inequality constraints must restrict $x$ to be in a convex subset of its domain, and any equality constraints must restrict $x$ to be in an affine subspace of its domain. Therefore, a convex optimisation problem in the form of equation \ref{eq:optimisation problem} is said to be a standard form convex optimisation problem if the objective function $f_0(x)$ is a convex function, all inequality constraint functions $f_i(x)$ are convex functions ($\forall i\in\{1, \hdots, m\}$), and all equality constraint functions $h_i(x)$ are affine functions ($\forall i\in\{1, \hdots, p\}$), in which case the problem is a convex optimisation problem. However, not all convex optimisation problems are in standard form, as the following example demonstrates:
\begin{equation}
\begin{aligned}
    \underset{x \in \mathbb{R}^2}{\text{Minimise}} \quad & x_1^2 + x_2^2 \\
    \text{Subject to} \quad & f_1(x) = \frac{x_1}{1 + x_2^2} \le 0 \\
    & h_1(x) = (x_1 + x_2)^2 = 0
\end{aligned} \label{eq:non standard convex problem}
\end{equation}
In this case, $f_1(x)$ is not a convex function, and $h_1(x)$ is not an affine function, therefore the problem is not a standard form convex optimisation problem, however the objective function is a convex function, and the feasible set defined by these constraints is a convex set:
\begin{align*}
    \frac{x_1}{1 + x_2^2} &\le 0 \\
    (\forall x_2 \in \mathbb{R}) \quad \frac{1}{1 + x_2^2} &> 0 \\
    \Rightarrow x_1 &\le 0 \\
    (x_1 + x_2)^2 &= 0 \\
    \Rightarrow x_1 + x_2 &= 0 \\
    \Rightarrow x_2 &= -x_1 \\
    \Rightarrow x_2 &\ge 0 \\
    \Rightarrow \mathcal{F} &= \{ x\in\mathbb{R}^2: x_1 \le 0 \quad \text{and} \quad x_2 = -x_1 \}
\end{align*}
Therefore the problem in equation \ref{eq:non standard convex problem} is equivalent to the following standard form convex optimisation problem (for which the primal optimal cost is clearly $0$, achieved when $x_1=x_2=0$):
\begin{align*}
    \underset{x \in \mathbb{R}^2}{\text{Minimise}} \quad & x_1^2 + x_2^2 \\
    \text{Subject to} \quad & f_1(x) = x_1 \le 0 \\
    & h_1(x) = x_1 + x_2 = 0
\end{align*}

\subsection{Hyperbolic Constraints And Second-Order Cones}
A second-order cone problem (SOCP) is an optimisation problem that has the following form (see \cite{boyd2004convex}, equation 4.36, page 156):
\begin{equation}
\begin{aligned}
    \underset{x \in \mathcal{X}}{\text{Minimise}} \quad & f^Tx \\
    \text{Subject to} \quad & \Vert A_ix + b_i \Vert_2 \le c_i^Tx + d_i \quad i\in\{1, \hdots, m\} \\
    & Fx = g
\end{aligned} \label{eq:SOCP}
\end{equation}
The following equivalence can be used to express several different types of problems as SOCPs, which holds for any $x \in \mathbb{R}^n$ and $y, z \in \mathbb{R}$ which satisfy $y>0$ and $z>0$:
\begin{equation}
    x^Tx \le yz \quad \Leftrightarrow \quad \left\Vert\begin{bmatrix}
        2x \\
        y - z
    \end{bmatrix}\right\Vert_2 \le y + z \label{eq:SOCP identity}
\end{equation}
This equivalence can be proved as follows:
\begin{align*}
    \left\Vert\begin{bmatrix}
        2x \\
        y - z
    \end{bmatrix}\right\Vert_2 &= \sqrt{\begin{bmatrix}
        2x \\
        y - z
    \end{bmatrix}^T \begin{bmatrix}
        2x \\
        y - z
    \end{bmatrix}} \\
    &= \sqrt{(2x)^T(2x) + (y-z)^2} \\
    &= \sqrt{4x^T x + y^2 - 2yz + z^2} \\
    0 \le \left\Vert\begin{bmatrix}
        2x \\
        y - z
    \end{bmatrix}\right\Vert_2 &\le y + z \\
    \Leftrightarrow \quad 0 \le \sqrt{4x^T x + y^2 - 2yz + z^2} &\le y + z \\
    \Leftrightarrow \quad 4x^T x + y^2 - 2yz + z^2 &\le y^2 + 2yz + z^2 \\
    \Leftrightarrow \quad 4x^T x &\le 4yz \\
    \Leftrightarrow \quad x^T x &\le yz
\end{align*}

\subsection{Support Functions}
\subsection{Largest-L Norm Of A Vector}
